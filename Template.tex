\documentclass[fleqn, 12pt]{article}

% packages %
\usepackage[headheight=110pt,margin=1in]{geometry} % page margins %
\usepackage{mathtools, amssymb} % math %
\usepackage{tabularx, multirow} % tables %
\usepackage{minted} % code %
\usepackage{graphicx} % graphics %
\usepackage{enumerate} % lists %
\usepackage{adjustbox} % images %
\usepackage[T1]{fontenc} % fonts %
\usepackage[protrusion=true,expansion=true]{microtype} % font clarity %
\usepackage{fancyhdr} % headers and footers %
\usepackage{lastpage} % reference page numbers %
\usepackage{color} % colors for code
\usepackage{tikz} % for graphs
\usepackage{soul} % for strikthroughout

% Document details %
\newcommand{\university}{<university>}
\newcommand{\name}{<name>}
\newcommand{\studentNumber}{<student id>}
\newcommand{\semester}{<semester>}
\newcommand{\assignmentType}{<type>}
\newcommand{\assignemntNumber}{<number>}
\newcommand{\dueDate}{<due date>}
\newcommand{\courseCode}{<course code>}
\newcommand{\courseTitle}{<course title>}

% Center image and diagrams %
\adjustboxset*{center}

% Prevent paragraph indents, this isn't an English assignment! %
\newlength\tindent
\setlength{\tindent}{\parindent}
\setlength{\parindent}{0pt}

\setminted{
    fontfamily=tt,
    gobble=0,
    frame=single,
    funcnamehighlighting=true,
    tabsize=4,
    obeytabs=false,
    mathescape=false
    samepage=false,
    showspaces=false,
    showtabs =false,
    texcl=false,
    breaklines=true,
}

% inline code
\definecolor{codegray}{gray}{0.9}
\newcommand{\code}[1]{\colorbox{codegray}{\texttt{#1}}}

% Code from tile
\newcommand{\codefile}{\inputminted}

% Graphing stuff
\usetikzlibrary{arrows.meta}
\usetikzlibrary{positioning}
\usetikzlibrary{matrix}
\usetikzlibrary{automata}

% Define ceiling and floor functions
\DeclarePairedDelimiter\ceil{\lceil}{\rceil}
\DeclarePairedDelimiter\floor{\lfloor}{\rfloor}

% Create set compliment command %
\newcommand{\setcomp}[1]{{#1}^{\mathsf{c}}}

% Create logic command aliases %
\newcommand{\limplies}{\rightarrow}
\newcommand{\nequiv}{\not\equiv}
\newcommand{\liff}{\leftrightarrow}

% Document header %
\newcommand{\makeheader}{
    \noindent
    \courseTitle \hfill \university\\
    \courseCode \hfill \semester
    \begin{center}
        \textbf{\assignmentType \#\assignemntNumber}\\
        \name \hspace{1pt} - \studentNumber\\
        \dueDate\\
    \end{center}
    \vspace{6pt}
    \hrule
}

% first page header & footer %
\fancypagestyle{firstpage}{
  \fancyhf{}
  \renewcommand{\footrulewidth}{0.1mm}
  \fancyfoot[R]{Assignment \assignemntNumber}
  \fancyfoot[C]{\thepage \hspace{1pt} of \pageref{LastPage}}
  \fancyfoot[L]{\courseCode}
  \renewcommand{\headrulewidth}{0mm}
}

% Page header and footers %
\fancyhf{}
\renewcommand{\footrulewidth}{0.1mm}
\fancyfoot[R]{Assignment \assignemntNumber}
\fancyfoot[C]{\thepage \hspace{1pt} of \pageref{LastPage}}
\fancyfoot[L]{\courseCode}
\fancyhead[L]{\name}
\fancyhead[R]{\studentNumber}

% Apply headers & footer page style %
\pagestyle{fancy}
\begin{document}

% ignore fancy headers on this page and use document header %
\thispagestyle{firstpage}
\makeheader

\section*{Assignment template}

Comes with variables for name and class info (lines 20-29). Auto generates page numbers, footers and headers. Supports many math and coding related packages.

\section*{Enumerations}

% simple listings, parameter can be changed to change bullet style
% use itemize for unordered lists
\begin{enumerate}[a)]
    \item item 
    \item item
    \item item
\end{enumerate}

\section*{Code}

%uses minted for formatting, only supports minted code
\codefile{javascript}{sample.js}

%no fancy formatting for inline code
\code{This is an example of inline code}

\section*{Graphics (png, pdf, jpg, etc...)}

% include any graphic type -- can change the width. I like to base
% it off of the text width
\adjincludegraphics[width=0.5\textwidth]{sample.jpg}

% add a page break
\newpage

\section*{Tikz graphs}

% uses tikz for graphing
\begin{center}
    \begin{tikzpicture}
    
        % defines all of the nodes within the picture
        \begin{scope}
            % node [type] number at location {contents};
            \node[state] (1) at (0,0) {-};
            \node[state] (2) at (3,2) {};
            \node[state] (3) at (6,0) {};
            \node[state] (4) at (9,2) {};
            \node[state] (5) at (12,0) {+};
        \end{scope}
        
        % defines all of the paths within the picture
        \begin{scope}[>={Stealth[black]}, every node/.style={fill=white}]
            %path [arrow] (start node) edge node {contents} (end node);
            \path [->] (1) edge[loop left] node {$a$} (1);
            \path [->] (1) edge node {$b$} (2);
            \path [->] (2) edge node {$a$} (3);
            \path [->] (2) edge[loop above] node {$b$} (2);
            \path [->] (3) edge[loop above] node {$a,b$} (3);
            \path [->] (3) edge node {$b$} (4);
            \path [->] (4) edge[loop above] node {$b$} (4);
            \path [->] (4) edge node {$a$} (5);
            \path [->] (5) edge[loop right] node {$a,b$} (5);
        \end{scope}
    \end{tikzpicture}
\end{center}

\section*{Math examples}

% I like to do math in an aligned environment
% To align things just do &=
$
    \begin{aligned}
        SQNR &= 6.02n + 1.76\\
        36 &= 6.02n + 1.76\\
        36 - 1.76 &=  6.02n\\
        34.24 &= 6.02n\\
        \frac{34.24}{6.02} &= 5.688
    \end{aligned}
$

\section*{Table example}

% A sample table which can be used to compute checksums
\begin{tabular}{r c c c}
      &    & E3 & 4F\\
      &    & 23 & 96\\
      &    & 44 & 27\\
    + &    & 99 & F3\\\hline
    = & 01 & E4 & FF\\
    + &    &    & 01\\\hline
    = &    & E5 & 00 \\\hline
    ' &    & 1A & FF \\
\end{tabular}\\\\

\section*{Advanced example of tabular and tikz picture to form tree}

\begin{tabular}{|c|c|}
    \hline
    Version 1 & Version 2\\\hline
    \begin{tikzpicture}
        \node{S} 
            child{
                node{S} 
                child{ 
                    node{S}
                    child{
                        node{b}
                    }
                }
                child{ 
                    node{bbb}
                }
            }
            child{
                node{bbb}
            };
    \end{tikzpicture}
    &
    \begin{tikzpicture}
        \node{S} 
            child{
                node{S} 
                child{ 
                    node{S}
                    child{
                        node{S}
                        child{
                            node{b}
                        }
                    }
                    child{
                        node{bb}
                    }
                }
                child{ 
                    node{bb}
                }
            }
            child{
                node{bb}
            };
    \end{tikzpicture}
    \\\hline
\end{tabular}\\\\

\end{document}